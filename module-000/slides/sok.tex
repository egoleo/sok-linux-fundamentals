\documentclass{beamer}
\usetheme{shadow}
\title{About Seed of Knowledge}
\author{Lorenzo Cabrini (lorenzo.cabrini@gmail.com)}
\date{17 October 2010}

\begin{document}
\frame{\titlepage}

\begin{frame}
\frametitle{Introducing Seed of Knowledge}
To start off, we want to give you a short introduction to the Seed of Knowledge method.
\end{frame}

\begin{frame}
\frametitle{There is a problem}
Get used to hearing "there is a problem". You will hear it a lot.

All Seed of Knowledge courses are problem-centered. Each module starts with a problem and ends with a solution to that problem.
\end{frame}

\begin{frame}
\frametitle{Real world scenarios}
The Seed of Knowledge method aims to teach you by exposing you to problems that you may encounter in the proverbial "real world".
\end{frame}

\begin{frame}
\frametitle{Understand the problem}
Before you attempt to solve any problem, make sure you fully understand it. Ask as many questions as you need to in order to clarify anything that you are not clear about.
\end{frame}

\begin{frame}
\frametitle{Nobody can solve complex problems}
Your ability to solve problems directly depends on your ability to break complex problems into smaller ones.
\end{frame}

\begin{frame}
\frametitle{Don't think, know}
You should acquire real knowledge. Guessing is not a good idea when a bad guess could potentially destroy important data.
\end{frame}

\begin{frame}
\frametitle{Experimenting}
Often you need to experiment in order to gain real knowledge. Do this only on copies of real data.
\end{frame}

\begin{frame}
\frametitle{Collaboration}
Collaborating with others is key to solving larger problems within reasonable time frames. The Seed of Knowledge method will let you practice working together with others on a team to solve problems.
\end{frame}

\begin{frame}
\frametitle{Be prepared}
It is important that you prepare ahead of every session. If you have been given materials to read before a session, make sure that you do so. 
\end{frame}
\end{document}
